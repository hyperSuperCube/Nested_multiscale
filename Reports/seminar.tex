\documentclass[aspectratio=169]{beamer}

\input{./template/macros.tex}

\usepackage{cancel}
\usepackage{listings}
\usepackage{xcolor}
\usepackage{tikz}
\usetikzlibrary{positioning, arrows}
% Define colors for code
\definecolor{codegreen}{rgb}{0,0.6,0}
\definecolor{codegray}{rgb}{0.5,0.5,0.5}
\definecolor{codepurple}{rgb}{0.58,0,0.82}
\definecolor{backcolour}{rgb}{0.95,0.95,0.92}

% C++ code style
\lstdefinestyle{cppstyle}{
	language=C++,
	backgroundcolor=\color{backcolour},   
	commentstyle=\color{codegreen},
	keywordstyle=\color{blue},
	numberstyle=\tiny\color{codegray},
	stringstyle=\color{codepurple},
	basicstyle=\ttfamily\footnotesize,
	breakatwhitespace=false,         
	breaklines=true,                 
	captionpos=b,                    
	keepspaces=true,                 
	numbers=left,                    
	numbersep=5pt,                  
	showspaces=false,                
	showstringspaces=false,
	showtabs=false,                  
	tabsize=2
}
\newif\ifmovies%
\moviestrue%
%\moviesfalse

\newif\ifsubs%
\substrue%
%\subsfalse

\newif\ifnotes%
\notestrue%
%\notesfalse

\usepackage{xfrac}
\usepackage{subfigure}
\usepackage{xcolor}	
\usepackage{stmaryrd}				%definir novas cores
\newcommand{\red}[1]{\textcolor{red}{#1}}
\newcommand{\blue}[1]{\textcolor{blue}{#1}}
\newcommand{\green}[1]{\textcolor{green}{#1}}

\usepackage{tikz}
\usetikzlibrary{tikzmark}
\usepackage[absolute,overlay]{textpos}

\begin{document}


%======================================================================
\begin{frame}[c]\frametitle{}

    \vspace{15mm}
    \begin{center}
    \Large{\textsc{
    Leverage Turbulence measuring and probing
    }}
    \end{center}
    \vspace{10mm}
    \begin{center}
    Hongkai Zheng, Zirui Wang
    \end{center}

    \vspace{0mm}
    \begin{center}
    April $11^{th}$ 2025
    \end{center}
\end{frame}

\begin{frame}[c]\frametitle{Goal}
	\textbf{\textcolor{red}{Assimilating Observed data into dynamical model to reconstruct high precision data}}
	\begin{columns}
		\begin{column}{0.5\textwidth}
			\begin{itemize}
				\item Review
				\begin{itemize}
					\item 2019 Stanford University + Sandia Lab using 4D-var to assimilate a jet flow with Re 13500 into LES to eliminate numerical sensitivity of turbulence
					\item 2022 Johns Hopkins University and Maryland University uses Ensemble-variational method to assimilate wall-pressure data in LES simulation under Mach 6
					\item 2024 Aoyama Gakuin University and Nagoya University using 2D PIV data to tune RANs model parameters under Mach 2 
				\end{itemize}
			\end{itemize}
		\end{column}
	\begin{column}{0.5\textwidth}
		\begin{itemize}
		\item Our Target:
		\begin{itemize}
			\item Aiming on Turbulence field reconstruction on Inertia range
			\item Proposing new Computational model and data assimilation approaches 
			\item Demonstrate the potential of precision reconstruction of flow field property in highly non-linear flow situations with coarse measurement.
			\item Make engineering compatible measuring tool chains
		\end{itemize}
	\end{itemize}
	\end{column}
	\end{columns}
\end{frame}
\begin{frame}[c]\frametitle{Innovations in CFD}
	\textbf{\textcolor{red}{Using nested multiscale to arrive at DNS accuracy while maintain a near LES cost}}
	\begin{itemize}
		\item Governing equation
		\begin{align*}
			\frac{\partial u}{\partial t}+(u\cdot{\nabla})u &= \nu\Delta u + f\\
			\nabla\cdot u &= 0
		\end{align*}
		\item Periodic boundary with Isotropic turbulence initialization (Rogallo)
		\item Nested multiscale phase 
		$$
		\begin{aligned}
			& \frac{\partial \boldsymbol{\theta}^\epsilon}{\partial t}+\left(\mathbf{u}^\epsilon \cdot \nabla\right) \boldsymbol{\theta}^\epsilon=\mathbf{0} \\
			& \left.\boldsymbol{\theta}^\epsilon\right|_{t=0}=\mathbf{x}
		\end{aligned}
		$$
		\begin{equation*}
			\boldsymbol{\theta}^\epsilon=\overline{\boldsymbol{\theta}}(t, \mathbf{x}, \tau)+\epsilon \widetilde{\boldsymbol{\theta}}(t, \overline{\boldsymbol{\theta}}, \tau, \mathbf{z})\quad \mathbf{z}=\frac{\overline{\boldsymbol{\theta}}}{\epsilon}, \quad \tau=\frac{t}{\epsilon},\quad \tilde{\boldsymbol{\theta}}(\boldsymbol{z}) = \tilde{\boldsymbol{\theta}}(\boldsymbol{z}+\boldsymbol{1})\quad \int\tilde{\boldsymbol{\theta}}(\boldsymbol{z}) d\boldsymbol{z} = 0
		\end{equation*}
		\item Soul: The multiscale structure is convected by mean flow and inducing cell (sub-grid) homogenization problem, which homogenizes mean flow.
	\end{itemize}
\end{frame}

\begin{frame}[c]\frametitle{Algorithm}
	\begin{itemize}
		\item Step 1. At t = 0 and $\tau = 0$, we have 
		\begin{equation*}
			\boldsymbol{\theta}_{i n t}=\mathbf{x}, \quad \mathbf{u}_{i n t}=\mathbf{U}, \quad \mathbf{w}_{i n t}=\mathbf{W}, \quad \mathbf{\Theta}_{\text {int }}=\mathbf{0}, \quad \mathcal{A}=\mathcal{I} .
		\end{equation*}
		\item Step 2. Solve cell problem for (w, q)
		\begin{columns}
			\begin{column}{0.7\textwidth}
				\begin{align*}
					\partial_\tau \mathbf{w}+D_z \mathbf{w} \mathcal{A} \mathbf{w}+\mathcal{A}^{\top} \nabla_z q-\frac{\nu}{\epsilon} \nabla \cdot\left(\mathcal{A} \mathcal{A}^{\top} \nabla_z \mathbf{w}\right)&=\mathbf{0}\\
					\left(\mathcal{A}^{\top} \nabla_z\right) \cdot \mathbf{w}&=0\\
					\left.\mathbf{w}\right|_{\tau=\tau_m}&=\mathbf{w}_{i n t}
				\end{align*}
			\end{column}
			\begin{column}{0.3\textwidth}
				\begin{align*}
					\begin{aligned}
						& \partial_\tau \boldsymbol{\Theta}+\left(\mathcal{I}+D_z \boldsymbol{\Theta}\right) D_x \overline{\boldsymbol{\theta}} \mathbf{w}=\mathbf{0} . \\
						& \left.\boldsymbol{\Theta}\right|_{\tau=t=0}=\boldsymbol{\Theta}_{\text {int }} .
					\end{aligned}
				\end{align*}
			\end{column}
		\end{columns}
		
		\item Step 3. Update large scale solution
				\begin{align*}
					\begin{aligned}
						\partial_t \mathbf{u}+\left(\mathbf{u} \cdot \nabla_x\right) \mathbf{u}+\nabla_x p+\nabla_x \cdot\left\langle[\mathbf{w} \otimes \mathbf{w}]_{\Delta}^*\right\rangle-\nu \Delta \mathbf{u}&=\mathbf{f} \\
						\nabla_x \cdot \mathbf{u}&=0\\
						 \left.\mathbf{u}\right|_{t=t_m}&=\mathbf{u}_{i n t}
					\end{aligned}
				\end{align*}
			
				\begin{align*}
					\begin{aligned}
						\partial_t \boldsymbol{\theta}+\left(\mathbf{u} \cdot \nabla_x\right) \boldsymbol{\theta}+\epsilon \nabla_x \cdot\left\langle[\boldsymbol{\Theta} \otimes \mathbf{w}]_{\Delta}^*\right\rangle&=\mathbf{0} \\
						 \left.\boldsymbol{\theta}\right|_{t=t_m}&=\boldsymbol{\theta}_{\text {int }}
					\end{aligned}
				\end{align*}
	\end{itemize}
\end{frame}
\begin{frame}
	\begin{itemize}
		\item Step 4. Go back to step 2 and start over
		\begin{equation*}
			\begin{aligned}
				\boldsymbol{\theta}_{\text {int }} & =\left.\boldsymbol{\theta}\right|_{t=t_{m+1}}, & \mathbf{u}_{i n t} & =\left.\mathbf{u}\right|_{t=t_{m+1}}, \quad \mathbf{w}_{i n t}=\left.\mathbf{w}\right|_{\tau=\tau_{m+1}}, \\
				\boldsymbol{\Theta}_{\text {int }} & =\left.\boldsymbol{\Theta}\right|_{\tau=\tau_{m+1}}, & \mathcal{A} & =\left.D_x \boldsymbol{\theta}\right|_{t=t_{m+1}} .
			\end{aligned}
		\end{equation*}
		\item Adaptive Technique: Solve the cell problem when needed
		\begin{itemize}
			\item The Jacobian of inverse flow map $D_x \boldsymbol{\theta}$ determines the significance of cell problem
			\begin{equation*}
				\begin{aligned}
					& \partial_\tau \mathbf{w}+\left(D_x \boldsymbol{\theta} \mathbf{w} \cdot \nabla_z\right) \mathbf{w}+D_x \boldsymbol{\theta}^{\top} \nabla_z q-\frac{\nu}{\epsilon} \nabla_z \cdot\left(D_x \boldsymbol{\theta} D_x \boldsymbol{\theta}^{\top} \nabla_z \mathbf{w}\right)=\mathbf{0}, \\
					& \left(D_x \boldsymbol{\theta}^{\top} \nabla_z\right) \cdot \mathbf{w}=0 \\
					& \left.\mathbf{w}\right|_{\tau=t=0}=\mathbf{W}(\mathbf{x}, \mathbf{z}) .
				\end{aligned}
			\end{equation*}
			\item Evaluate $G=\left\|\left(D_x \theta\right)_n^T\left(D_x \theta\right)_n-I\right\|$ at time t = n as the standard
		\end{itemize}
		\item Strength: No parameter tuning, DNS comparable accuracy. $\epsilon$ determines the background mesh density.
	\end{itemize}
\end{frame}

\begin{frame}[c]\frametitle{Result}
	\begin{figure}[!htb]
		\begin{minipage}{0.49\textwidth}
			\centering
			\includegraphics[width=1\linewidth]{Figures/spec1.png}
			
		\end{minipage}\hfill
		\begin{minipage}{0.49\textwidth}
			\centering
			\includegraphics[width=1\linewidth]{Figures/spec2.png}
			
		\end{minipage}
	\end{figure}
\end{frame}

\begin{frame}[c]\frametitle{Innovations in Data assimilation}
	
\end{frame}

\begin{frame}[c]\frametitle{Algorithm}
	
\end{frame}

\begin{frame}[c]\frametitle{Result}
	
\end{frame}

%======================================================================

%======================================================================
\begin{frame}\frametitle{}

\vspace*{0.2in}

\begin{center}

\includegraphics[width=0.5\textwidth]{./template/smartAPI-logo-2.pdf}

\vspace*{0.35in}
\cPI{\huge Questions?}

\vspace*{0.5in}
\begin{minipage}{0.8\textwidth}
Thanks!!!!!!!!!!!!!
\end{minipage}

\end{center}


\end{frame}
%======================================================================



%======================================================================


\end{document}
